% 2015 | IT-Syndikat
% vim: set spell tw=79 :

\documentclass{itsarticle}
\title{Clean Windows 7 Installation}
\author{Alex ``W4RH4WK'' Hirsch}
\date{2015-06-04}

\begin{document}

\maketitle

\section*{Abstract}
\label{sec:abstract}

This manual will guide you through a clean Windows 7 installation process. It
will also focus on creation of the installation medium as well as installing
common software. Most recommendations described in this manual are personal
preferences of the author. Even though this manual is aimed to install Windows
on a virtual machine, it can also be used for installations on a \emph{real}
machine.

\section{Crafting an Installation Image}
\label{sec:install_image}

For our clean installation we will create an installation image which already
contains the service pack 1 and windows updates. Doing this will speedup the
whole installation process.

\subsection{Getting the base Installation Image / Medium}
\label{ssec:win7_base_iso}

As starting point a Windows 7 installation Image or Disk is required. It
should not be that hard to find one. 64 bit Ultimate edition is recommended,
although this should work for 32 bit and other editions too.

The following process will require you to ether insert the disk in your disk
drive or to mount (or extract) the obtained image file.

\subsection{Acquiring Windows Updates}
\label{ssec:get_win7_updates}

Of course in order to integrate windows updates (and service packs) into your
installation medium you first have to download these updates. A tool going by
the name of Windows Updates Downloader\footnotemark\ will take care of this.
Download and launch this program.

\footnotetext{\url{http://www.windowsupdatesdownloader.com}}

You will have to enter a ``Update List'' to let the program know for what
installation you want to get the updates. We will use two lists in this manual:

\begin{itemize}
    \item Windows 7 Service Pack 1\footnotemark
    \item Windows 7 Updates Extra Updates ULZ\footnotemark
\end{itemize}

\footnotetext{listed here: \url{http://www.windowsupdatesdownloader.com/UpdateLists.aspx}}
\footnotetext{
    64 Bit: \url{https://www.raymond.cc/blog/download/did/3131/}\\
    32 Bit: \url{https://www.raymond.cc/blog/download/did/3132/}
}

Go ahead and download \emph{all} updates offered by these two lists to a
dedicated download folder.

\subsection{Combine Base Installation and Updates}
\label{ssec:ntlite}

The are multiple ways of doing this, the one I commonly use is by using a tool
called NTLite\footnotemark. This tool is free for personal use and allows you
to create a new installation image based on an already existing one plus
integrating downloaded windows updates.

\footnotetext{\url{https://www.ntlite.com}}

Before starting the repackaging process you may want to set the programs temp
folder to some place offering enough space. You can find the corresponding
option in \menu{File > Settings > Temporary files directory}. Restart NTLite
after changing the temp directory.

Although NTLite offers a lot of features even in the free version, I do not
recommend altering the installation image in any way except for the update
integration.

At this point you may also include additional driver into the image if needed,
NTLite supports this.

Extract the original Windows 7 installation image so NTLite can write to it.
After adding the extracted file location in NTLite you should be able to select
and load the desired edition.

% TODO
After adding the mounted / extracted image in NTLite the contained editions
should be listed under ``Image History''.

\subsection{Installation Medium}
\label{ssec:medium}

The resulting installation image will be quite big compared to the base image,
therefore you should choose to create a bootable USB drive if you plan to
install Windows 7 on a dedicated computer. A virtual machine will boot the
image just fine.

\subsubsection{Setup USB drive}

\begin{enumerate}
    \item grab a USB drive which fits your installation image
    \item backup the data on the USB drive
    \item format the USB drive with NTFS (``quick format'' is okay)
    \item mount / extract your newly crafted installation image
    \item run \texttt{DISKPART} from a command prompt
    \item use \texttt{LIST VOL} to list attached volumes
    \item use \texttt{SELECT VOL} to select the USB drive
    \item use \texttt{ACTIVE} to mark the volume as active
    \item use \texttt{EXIT} to quit \texttt{DISKPART}
    \item run \texttt{BOOTSECT /NT60 x:} located inside the \texttt{BOOT}
        folder on your installation image where \texttt{x:} is the drive letter
        of the USB drive.
    \item lastly copy \emph{all} data from the installation image to the USB
        drives' root folder
\end{enumerate}

Your USB drive should now be bootable and can be used to run your windows 7
installation image.

\section{Installation Process}
\label{sec:installation}

This section will go through the base installation of Windows 7. Here we will
use a virtual machine utilising VirtualBox\footnotemark.

\footnotetext{\url{http://virtualbox.org}}

\subsection{Setting up the Virtual Machine}
\label{ssec:virtual_box}

Fire up VirtualBox and create a new VM, I usually do following adjustments:

\begin{itemize}
    \item a minimum of \SI{1024}{\mega\byte} of memory
    \item a minimum of \SI{50}{\giga\byte} of hard drive storage
    \item a minimum of \SI{64}{\mega\byte} of video memory
    \item enable 3D acceleration
    \item enable 2D video acceleration
    \item set network adapter to ``bridged'' if ``NAT'' results in connection
        problems
\end{itemize}

Do not forget to insert your installation image into the virtual drive.

\subsection{Base Installation}
\label{ssec:base_install}

\begin{enumerate}
    \item leave language and time format on ``English (United States)'' for now
    \item you may change the keyboard layout if you are not comfortable with US
        keyboard
    \item select ``Windows 7 Ultimate'' when prompted
    \item click ``Custom (advanced)'' on the next screen
    \item select the desired disk for your system
\end{enumerate}

Next click through the setup assistant:

\begin{enumerate}
    \item click skip on the ``Windows product key'' page
    \item click ``Ask me later'' on the next screen
    \item click ``Public network'' on the ``current location'' page
\end{enumerate}

After your installation booted for the first time, add / remove the following
features:

\menu{Control Panel > Programs and Features > Turn Windows features on or off}

\begin{itemize}
    \item uncheck ``Games''
    \item uncheck ``Internet Explorer''
    \item uncheck ``Media Features''
    \item uncheck ``Tablet PC Components''
    \item check ``Telnet Client''
    \item uncheck ``Windows Gadget Platform''
\end{itemize}

\textit{If you use Internet Explorer to download additional software (see
\cref{sec:software}) do not uncheck the corresponding bullet.}

Click ``OK'' and reboot.

\subsection{Additional Windows Updates}
\label{ssec:updates_post_install}

Install \emph{all} updates available next through Windows Updates, reboot after
the update process. Rinse and repeat until no more updates are available.

\textit{You can skip language packs and updates related to Windows Defender and
Microsoft Security Essentials if you want.}

\section{Drivers}
\label{sec:drivers}

Install all drivers required by your system, most of them should have been
downloaded by Windows Updates automatically. Check the device manager (see
\cref{ssec:shortcuts}) to see if there are still components not recognized by
your system.

\section{Fine Tuning}
\label{sec:fine_tuning}

\subsection{Shortcuts}
\label{ssec:shortcuts}

Following shortcuts may help you access settings more quickly:

\begin{description}
    \item[Control panel] \hfill \\
        hit \keys{Win + r} enter ``control''

    \item[Services] \hfill \\
        hit \keys{Win + r} enter ``services.msc''

    \item[Computer Management] \hfill \\
        hit \keys{Win + r} enter ``compmgmt.msc''

    \item[Advanced system settings] \hfill \\
        hit \keys{Win + pause} select ``Advanced system settings''

    \item[Device Manager] \hfill \\
        hit \keys{Win + r} enter ``devmgmt.msc''
\end{description}

This will probably the most important part of the whole setup process. In here
we will fine tune various windows settings to fit a power users needs. Again
these are my personal preferences, adjust things to your needs.

\subsection{Disabling Services}
\label{ssec:services}

Disabling certain system services may speedup your system significantly but may
also render some features useless, these are my preferences:

% TODO

\section{Additional Software}
\label{sec:software}

Of course you can install the software you need the way you want to. But there
are 2 ways I recommend for installing basic components.

\begin{itemize}
    \item Ninte\footnotemark
    \item chocolatey\footnotemark
\end{itemize}

\footnotetext{\url{https://ninite.com}}
\footnotetext{\url{https://chocolatey.org}}

My basic components include:

\begin{itemize}
    \item 7-Zip
    \item Firefox
    \item Notepad++
    \item SysinternalsSuit
    \item VLC
    \item WinDirStats
    \item qBittorrent
\end{itemize}

\section{Activation}
\label{sec:activation}

Active Windows 7 now using your favorite method of activation.

\section{VirtualBox Guest Additions}
\label{sec:guest_additions}

Do not forget to install guest additions if you are using a virtual machine.

\section{Maintenance}
\label{sec:maintenance}

Lastly, \textbf{always keep your system clean!} If you are unsure what a
software will do to your setup, run it inside a virtual machine and take
advantage of snapshots.

In addition you may use the Disk Cleanup assistant Windows brings along to
remove leftover data from temporary files or \textbf{Windows Updates}.
WinDirStats shows you a layout of all files located on a disk sorted by
filesize. One could also install CCleaner to keep application data and registry
clean.

\textit{I do not recommend installation updates automatically, to this manual
whenever time allows. Include optional updates but always be aware that Windows
Updates may include stuff you do not want, like Bing Desktop, Skype,
Silverlight, \ldots}

\end{document}
